\documentclass[10pt]{article}
\usepackage[landscape,a4paper,centering,twocolumn,
textwidth=290mm,textheight=205mm]{geometry}
\usepackage{fontspec}
\usepackage{polyglossia}
\usepackage[reqno]{amsmath}
\usepackage{amssymb,amsthm,stmaryrd}
\usepackage{unicode-math}
\usepackage{paralist}
\usepackage{color,bbding}

\pagenumbering{gobble}
\setlength\columnseprule{0.4pt}
\setdefaultlanguage{french}

\newcommand\R{\mathbb{R}}
\newcommand\N{\mathbb{N}}
\newcommand\K{\mathbb{K}}
\newcommand\abs[1]{\left\lvert#1\right\rvert}
\newcommand\norm[1]{\left\lVert#1\right\rVert}
\newcommand\val[1]{\textcolor{blue}{#1}}
\newcommand\att[1]{\textcolor{red}{\FourStar}\textcolor{blue}{#1}}

\theoremstyle{plain}
\newtheorem{thm}{Théorème}
\newtheorem{cor}[thm]{Corollaire}
\newtheorem{lem}[thm]{Lemme}
\newtheorem{prop}[thm]{Proposition}
\newtheorem{propdef}[thm]{Proposition \& définition}

\theoremstyle{definition}
\newtheorem{defn}{Définition}

\theoremstyle{remark}
\newtheorem{rem}{Remarque}
\newtheorem*{xrem}{Remarque}
\newtheorem{ex}[rem]{Exemple}
\newtheorem{meth}[rem]{Méthode}

\begin{document}
\begin{center}
    \Large\textbf{Real Time Computing}
\end{center}
\begin{itemize}
    \item System that reacts to exteriors events. Events are
        asynchronous regarding the activities carried out by the computer.
    \item Every events must be treated, real time systems or reactive systems.
    \item The system must react in a specified delay.
    \item Determinism is compulsory, each operation must be done in a defined
        amount of time.
\end{itemize}

\section{Fundamentals}
\subsection{Real Time System}
\begin{defn}
    A real time system is a system which reacts in a foreseeable time to each
    external stimuli which reaches it.

    Processing time must not depend on the input data.
\end{defn}
\begin{rem}
    Use of buffers endangers determinism, for instance, computing time won't be
    the same if data has been buffered.
\end{rem}

\subsection{RTS taxonomy}
Three types of rts:
\begin{itemize}
    \item Hard real-time: critical system, every deadline must be treated, in 
        the given amount of time.
    \item Soft real-time: some deadlines can be missed.
\end{itemize}

\paragraph{Lifetime of data}
Data must be processed quickly enough, as they might be transient (e.g.\ gnss
location while moving). To manage data, they are timestamped. This timestamp is
examined before processing data to verify whether the data is relevant.

\subsection{Time factor}
Application based real time systems 
\begin{itemize}
    \item Delivery time: max amount of time allowed to carry out the associated
        processing to an event.
    \item Urgency: Require to impose a priority hierarchy taking into account
        simultaneous event requests while insuring that each event could be
        dealt with within its own time limit (delivery time). Enables conceptor
        to set priorities to tasks.
\end{itemize}

\subsection{Dealing with the Timing Constraints}
\begin{description}
    \item [Low timing constraint]
        \begin{itemize}
            \item Process evolving relatively slowly (compared to the system).
            \item Use of a sampling loop, in which  each sensor is queried.
            \item Looping frequency must be controlled (worst case).
        \end{itemize}
    \item[Low timing constraint overall and  constraining events]
        \begin{itemize}
            \item Use of interruption mechanism.
            \item Still using a main sampling loop, interrupted by interrupt
                requests.
        \end{itemize}
    \item[High timing constraint]
        \begin{itemize}
            \item Processing not correlated to event occurrence.
            \item Event logged, program manages the processing sequence.
        \end{itemize}
\end{description}

\subsection{Real time application}
Multi tasking, with more specifications for each task.
\begin{itemize}
    \item Priority level
    \item Behaviour, some allows certification
        \begin{itemize}
            \item periodic (with time constant period)
            \item not periodic
            \item occasional (sporadic)
        \end{itemize}
    \item Hardware and software resources,
    \item Temporal specifications, dates and delays, four dates
        \begin{itemize}
            \item min beginning, resources not available yet
            \item max beginning: if task starts after this date, program
                will end too late
            \item min end, result can be delivered before this date, but end
                user not available
            \item max end
        \end{itemize}
\end{itemize}

\begin{defn}[Preemptive scheduling]
    Kernel can suspend any task at any time to run another task (e.g.\ with
    higher priority).
\end{defn}

\begin{description}
    \item[immediate tasks]
        \begin{itemize}
            \item high priority
            \item process events and operates actions
            \item critical tasks
        \end{itemize}
    \item[deferred tasks]
        low priority tasks
\end{description}

\subsection{Speed v deadline requirements}

\section{Technology}

\subsection{Real time executives}
Multi purpose os
\begin{itemize}
    \item fair scheduling between tasks
    \item increasing number of apps, swapping\dots
    \item buffering not relevant: no determinism while loading data
\end{itemize}

\paragraph{Kernel, executive and operating system concepts}
\begin{itemize}
    \item RT kernel
        \begin{itemize}
            \item core of kernel manages bare minimum.
            \item less intermediates
        \end{itemize}
    \item RT executive
        \begin{itemize}
            \item higher set of a rt kernel (IO management, memory management,
                specification and programming utilities, debugging tools)
        \end{itemize}
    \item RTOS
        \begin{itemize}
            \item higher set of a real time executive
        \end{itemize}
\end{itemize}

\paragraph{Scheduling}
\begin{itemize}
    \item Programme managing the sequence of tasks
    \item Multi purpose os $\ne$ rt
    \item Sequencing modified only if new events
    \item Multi purpose scheduling policies
        \begin{itemize}
            \item FIFO, processes in queue. No reactivity
            \item Round-robin, each task is executed during a specified time
                (when FIFO executes the whole task), used in multi purpose os.
        \end{itemize}
    \item Real time scheduling policies
        \begin{itemize}
            \item Fixed priority, priorities can't change (chosen at development
                time) (small systems)
            \item RMS, periodic tasks, allows formal proofs on
                sequencing, imposes priorities considering frequency (bigger
                systems)
            \item EDF Earliest Deadline First, not fixed priorities: can be
                modified by sequencer, priority higher if deadline closer, needs
                more computation time
        \end{itemize}
\end{itemize}
\begin{rem}
    The more a kernel is preemptive, the more it is reactive.
\end{rem}

\section{Simulation}
\end{document}
