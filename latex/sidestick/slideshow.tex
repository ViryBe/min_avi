\documentclass[footheight=2em]{beamer}
\usetheme[hideothersubsections]{Hannover}
\usepackage[T1]{fontenc}
\usepackage[utf8]{inputenc}
\usepackage{amssymb}
\usepackage{amsthm}
\usepackage{amsmath}
\usepackage{color,graphicx}
\usepackage{gensymb}
\usepackage{tikz}

\usecolortheme{dolphin}

\title{Projet avion mineure AVI}
\author{Damien Thoral, Gabriel Hondet, Benoit Viry, Nicolas Soulard}
\date{\today}

% ----------------------------------------------------------
% nuremotation des pages -----------------------------------
% ----------------------------------------------------------
\def\swidth{1.6cm}
\setbeamersize{sidebar width left=\swidth}
\setbeamertemplate{sidebar left}
{%
  {\usebeamerfont{title in sidebar}
    \vskip1.5em
    \usebeamercolor[fg]{title in sidebar}
    \insertshorttitle[width=\swidth,center,respectlinebreaks]\par
    \vskip1.25em
  }
  {
    \usebeamercolor[fg]{author in sidebar}
    \usebeamerfont{author in sidebar}
    \insertshortauthor[width=\swidth,center,respectlinebreaks]\par
    \vskip1.25em
  }
  \hbox to2cm{\hss\insertlogo\hss}
  \vskip1.25em
  \insertverticalnavigation{\swidth}
  \vfill
  \hbox to2cm{\hskip0.6cm\usebeamerfont{subsection in
      sidebar}\strut\usebeamercolor[fg]{subsection in
      sidebar}\insertframenumber /\inserttotalframenumber\hfill}
  \vskip3pt
}
% ----------------------------------------------------------


\begin{document}

\frame{\titlepage}

\AtBeginSection[]
{%
  \begin{frame}
    \frametitle{Plan}
    \tableofcontents[currentsection]
  \end{frame}
}

\section{Spécifications}

\section{Boucle de scrutation}
\begin{frame}[t]{Boucle de scrutation}
  \begin{block}{Déclenchement}
    Lancée a chaque réception d'un message de la part de l'auto pilote.
  \end{block}

  \begin{block}{Contenu}
    \begin{enumerate}
      \item mise a jour état du mini manche (actif ou non)
      \item si actif, lecture des données provenant du mini manche
      \item vérification de la valeur
      \item revoit de la valeur sur le bus.
    \end{enumerate}
  \end{block}

  \begin{block}{Distinction des grandeurs}
    Une boucle ne concerne \emph{qu'une seule} grandeur, déterminée par le
    message la déclenchant.
  \end{block}
\end{frame}

\section{Gestion des événements entrants}
\subsection{Module \texttt{pygame}}
\begin{frame}[t]{Module \texttt{pygame}}
  \begin{block}{File d'événements}
    Une file contenant tous les événements provenant du mini manche est
    maintenue par \texttt{pygame}.
  \end{block}
  \begin{block}{Interaction avec la file}
    \begin{itemize}
      \item \texttt{get()} revoit les données du minimanche et vide la file.
      \item \texttt{get(evt)} ne traite que les événements du type \texttt{evt}.
    \end{itemize}
  \end{block}
\end{frame}

\subsection{Gestion des événements}
\begin{frame}[t]{Gestion de la file}
  \begin{block}{Données}
    Chaque grandeur dispose
    \begin{itemize}
      \item d'une liste contenant toutes les entrées depuis le dernière boucle
        la concernant,
      \item une variable \texttt{LASTAXISV} contenant la dernière
        valeur relevée.
    \end{itemize}
  \end{block}
  \begin{block}{Obtention des données}
    \begin{enumerate}
      \item La file d'événements est vidée dans une liste temporaire,
      \item les événements de la file temporaire sont repartis entre les listes
        correspondant aux grandeurs,
      \item la variable \texttt{LASTAXISV} correspondante est modifiée, et sa
        valeur est renvoyée.
    \end{enumerate}
  \end{block}
\end{frame}

\begin{frame}[t]{Gestion de l'auto pilot}
  \begin{block}{Besoin de thread}
    L'appui de boutons pour désactiver l'auto pilote devant être effectif a tout
    moment, un thread dédie est nécessaire.
  \end{block}
  \begin{block}{Contenu du thread}
    Boucle infinie appelant la procédure \texttt{update\_ap},
    \begin{enumerate}
      \item récupérer l'état du pilote automatique via une variable
        globale,
      \item récolter des événements de la file concernant les appuis boutons,
      \item si un bouton a été appuyé et que le pilote n'était pas enclenche,
        alors desenclencher le pilote automatique,
      \item renvoyer l'état de l'auto pilote.
    \end{enumerate}
  \end{block}
\end{frame}

\subsection{Traitement des données}
\begin{frame}[t]{Plages de valeurs}
  \begin{block}{Définition des plages}
    \begin{itemize}
      \item \(n_z \in \mathbf{D}_{n_z} = [-1, 2.5]\)m\(\cdot\)s\(^{-2}\),
      \item \(p \in \mathbf{D}_p = [-15, 15]\deg\).
    \end{itemize}
  \end{block}
  \begin{block}{Conformité des données}
    \begin{itemize}
      \item Si auto pilote activé, les données passent par une
        fonction \(f\colon D \mapsto \mathbf{D}_{n_z}\) ou \(\mathbf{D}_p\),
      \item sinon, une fonction \(g\colon[-1, 1] \mapsto \mathbf{D}_{n_z}\) ou
        \(\mathbf{D}_p\) associe a la valeur en sortie du mini manche une valeur
        adéquate.
    \end{itemize}
  \end{block}
\end{frame}


\section{Tests}
\begin{frame}[t]{Définition des tests}
  \begin{block}{Valeurs de test}
    Envoi d'une série de 10 valeurs aléatoires au mini manche,
    \begin{itemize}
      \item \(n_z \in [-5, 5]\),
      \item \(p \in [-30, 30]\).
    \end{itemize}
  \end{block}
  \begin{block}{Test de valeurs}
    Vérification de la conformité des valeurs en sortie,
    \begin{itemize}
      \item \(n_z \in [-1, 2.5]\),
      \item \(p \in [-15, 15]\).
    \end{itemize}
  \end{block}
  \begin{block}{Boucle temporelle}
    Envoi d'une valeur de \(n_z\) et de \(p\) à chaque boucle temporelle.
  \end{block}
\end{frame}

\begin{frame}[t]{Résultats des tests}
  \begin{block}{Valeur envoyée dans l'intervalle admissible}
    \begin{enumerate}
      \item Comparaison entre la valeur envoyée et la valeur recue,
      \item \texttt{True} si les deux valeurs sont identiques, \texttt{False}
        sinon.
    \end{enumerate}
  \end{block}
  \begin{block}{Valeur envoyée inférieure a la valeur minimale (resp.\
      maximale) admissible}
    \begin{enumerate}
      \item Comparaison valeur reçue et valeur minimale (resp.\ maximale)
        admissible,
      \item \texttt{True} si égalité, \texttt{False} sinon.
    \end{enumerate}
  \end{block}
  \begin{block}{Saturation}
    L'orque les valeurs sont hors de l'intervalle admissible, le mini manche doit
    renvoyer la borne la plus proche de l'intervalle.
  \end{block}
\end{frame}
\end{document}
