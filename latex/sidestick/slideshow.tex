\documentclass[tikz, footheight=2em]{beamer}
\usetheme[hideothersubsections]{Hannover}
\usepackage[T1]{fontenc}
\usepackage[utf8]{inputenc}
\usepackage{amssymb}
\usepackage{amsthm}
\usepackage{amsmath}
\usepackage{color,graphicx}
\usepackage{gensymb}
\usepackage[]{tikz}
\usetikzlibrary{quotes,angles}
\usepackage{tikz}
\usetikzlibrary{arrows}
\usetikzlibrary{calc}
\usetikzlibrary{positioning}

\usecolortheme{dolphin}

\title{Projet avion mineure AVI}
\author{Damien Thoral, Gabriel Hondet, Benoit Viry, Nicolas Soulard}
\date{}

% ----------------------------------------------------------
% nuremotation des pages -----------------------------------
% ----------------------------------------------------------
\def\swidth{1.6cm}
\setbeamersize{sidebar width left=\swidth}
\setbeamertemplate{sidebar left}
{%
  {\usebeamerfont{title in sidebar}
    \vskip1.5em
    \usebeamercolor[fg]{title in sidebar}
    \insertshorttitle[width=\swidth,center,respectlinebreaks]\par
    \vskip1.25em
  }
  {
    \usebeamercolor[fg]{author in sidebar}
    \usebeamerfont{author in sidebar}
    \insertshortauthor[width=\swidth,center,respectlinebreaks]\par
    \vskip1.25em
  }
  \hbox to2cm{\hss\insertlogo\hss}
  \vskip1.25em
  \insertverticalnavigation{\swidth}
  \vfill
  \hbox to2cm{\hskip0.6cm\usebeamerfont{subsection in
      sidebar}\strut\usebeamercolor[fg]{subsection in
      sidebar}\insertframenumber /\inserttotalframenumber\hfill}
  \vskip3pt
}
% ----------------------------------------------------------


\begin{document}

\frame{\titlepage}

\AtBeginSection[]
{%
  \begin{frame}
    \frametitle{Plan}
    \tableofcontents[currentsection]
  \end{frame}
}

\section{Specifications}

\section{Boucle de scrutation}
\begin{frame}[t]{Boucle de scrutation}
  \begin{block}{Declenchement}
    Lancee a chaque reception d'un message de la part de l'auto pilote.
  \end{block}

  \begin{block}{Contenu}
    \begin{enumerate}
      \item mise a jour etat du minimanche (actif ou non)
      \item si actif, lecture des donnees provenant du minimanche
      \item verification de la valeur
      \item renvoit de la valeur sur le bus.
    \end{enumerate}
  \end{block}

  \begin{block}{Distinction des grandeurs}
    Une boucle ne concerne \emph{qu'une seule} grandeur, determinee par le
    message la declenchant.
  \end{block}
\end{frame}

\section{Gestion des evenements entrants}
\subsection{Module \texttt{pygame}}
\begin{frame}[t]{Module \texttt{pygame}}
  \begin{block}{File d'evenements}
    Une file contenant tous les evenements provenant du joystick est maintenue
    par \texttt{pygame}.
  \end{block}
  \begin{block}{Interaction avec la file}
    \begin{itemize}
      \item \texttt{get()} renvoit les donnees du joystick et vide la file.
      \item \texttt{get(evt)} ne traite que les evenements du type \texttt{evt}.
    \end{itemize}
  \end{block}
\end{frame}

\begin{frame}[t]{Gestion de la file}
  \begin{block}{Donnees}
    Chaque grandeur dispose d'une liste contenant toutes les entrees depuis le
    derniere boucle la concernant.
  \end{block}
\end{frame}

\begin{frame}[t]{Le besoin d'un thread}
  \begin{block}{\texttt{}}
    Pour capturer les evenements \texttt{JOYBUTTONDOWN} et détecter quand le
    pilotage automatique doit être déconnecter, il a été necéssaire de d'implémenter
    un \texttt{thread} spécifique.
  \end{block}
\end{frame}


\section{Tests}
\end{document}
